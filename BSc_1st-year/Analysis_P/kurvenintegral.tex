\section{Kurvenintegral (Linienintegral)}
\subsection{1. Art}
Das Wegintegral einer \underline{stetigen Funktion} $f: \R^n \to \R$ entlang
eines stetig differenzierbaren Weges $\gamma: [a,b] \to \R^n$ ist definiert durch:
\[
\int_\gamma f ds := \int_a^b f(\gamma(t)) \|\gamma(t)'\|_2 dt
\]

\underline{Euklidische Norm}: $\|\vec{a}\|_2 = \sqrt{a_x^2 + a_y^2 + \ldots}$.
Achtung: Beim Integral muss man zuerst $\gamma(t)$ nach $t$ ableiten und erst
dann die Norm davon berechnen!

\subsubsection{Beispiel}
\todo{\ldots}

\subsection{2. Art}
Das Wegintegral über ein \underline{stetiges Vektorfeld} $\vec{f}: \R^n \to \R^n$
entlang eines stetig differenzierbaren Weges $\gamma: [a,b] \to \R^n$ ist definiert
durch:
\[
\int_\gamma \vec{f}(\vec{x}) d\vec{x} := \int_a^b \left< \vec{f}(\gamma(t)), \gamma(t)' \right> dt
\]

\underline{Skalarprodukt}: $\left< \vec{a}, \vec{b} \right> = a_x b_x + a_y b_y + \ldots$

\subsubsection{Beispiel}
\todo{\ldots}

\subsection{Parametrisierung von Kurven}
Grundlägender Tipp: Skizze machen, um Grenzen und Kurve besser zu verstehen und
schneller auf die Parametrisierung zu kommen.

\begin{itemize}[leftmargin=*]
	\item Wenn die Kurve in der Form
	\[
	C = \{\vec{r} \in \R^n | \vec{r} = \gamma(t), a \leq t \leq b\}
	\]
	bereits gegeben, so ist klar, dass $\gamma(t)$ der Weg ist und das Integral von
	$a$ nach $b$ verläuft.
	
	\item Die Paramtrisierung einer \underline{Strecke} von $\vec{a}$ nach $\vec{b}$:
	$\gamma(t) = \vec{a} + t(\vec{b}-\vec{a}), \quad 0 \leq t \leq 1$
	
	\item Die Parametrisierung eines \underline{Kreises} mit Mittelpunkt $(x_0, y_0)$ und
	Radius $r$ ist: $\gamma(t) =
	\begin{pmatrix}
	x_0 + r \cos(t)\\
	y_0 + r \sin(t)
	\end{pmatrix}$. Für einen vollen Kreis gilt $0 \leq t \leq 2\pi$, für Kreisteile
	schränkt man diesen Intevall entsprechend ein.
	
	\item Parametrisierung eines \underline{Graphen} der Funktion $f(x)$ für $x$
	zwischen $a$ und $b$: $\gamma(t) =
	\begin{pmatrix}
	t\\
	f(t)
	\end{pmatrix}, \quad a \leq t \leq b$
\end{itemize}

\subsection{Berechnung}
Die Berechnung findet in drei Schritten statt:
\begin{enumerate}[leftmargin=*]
	\item Parametrisierung der Kurve $C$ als $C = \{\vec{r} \in \R^n | \vec{r} = \gamma(t), a \leq t \leq b\}$
	\item Einsetzen ins Integral: $\int_C f(\vec{r})\,ds = \int_a^b f(\gamma(t)) \|\gamma(t)'\|\,dt$.
	Man setzt also für die Variabeln von $f$ die Komponenten von $\gamma$ ein und
	multipliziert dies dann mit dem Betrag der Ableitung nach $t$ von $\gamma$.
	\item Integral ausrechnen.
\end{enumerate}
