\section{Differenzierbarkeit}
\subsection{Defintion}
$f$ ist in $a \in I$ differenzierbar mit der Ableitung $f'(a)$, wenn
\[
\lim_{x \to a} \frac{f(x) - f(a)}{x - a} =: f'(a) = \frac{d}{dx}f(a)
\]
existiert.

Ist $f'$ stetig im Definitionsbereich, so heisst $f$ stetig differenzierbar. Man
kann also $f$ differenzieren und bekommt mit $f'$ eine stetige Funktion. Es gilt
auch $f \in C^1(I)$. $C^n(I)$ ist die Menge der $n$-mal stetig differenzierbaren
Funktionen über dem Intervall $I$.

\subsection{Mittelwertsatz (Satz von Lagrange)}
Ist $f$ au $[a,b]$ stetig und in $]a, b[$ differenzierbar, so gibt es ein $c
\in ]a,b[$ mit
\[
\frac{f(b) - f(a)}{b-a} = f'(c)
\]

\subsubsection{Beispiel: Ungleichungen}
Zu zeigen: $e^x(y-x) < e^y - e^x < e^y(y-x), \; \forall x < y$:


Der Mittelwertsatz wird auf die Exponentialfunktion angewendet. Damit gilt für
ein Paar von Zahlen $x < y$ ein $u \in ]x,y[$, für welches gilt: $\frac{e^y
- e^x}{y-x} = e^u$. Weil die Exponentialfunktion in $\R$ streng monoton wachsend
ist, gilt $e^x < e^u <^y$ und somit gilt: $e^x < \frac{e^y-e^x}{y-x} < e^y$.
Multipliziert man nun mit dem Nenner des Bruchs bekommt man: $e^x (y-x) < e^y -
e^x < e^y(y-x)$.

\subsection{Monotonie}
\begin{itemize}
	\item $f' > 0 \Rightarrow f$ streng monoton steigend
	\item $f' \geq 0 \Leftrightarrow f$ monoton steigend
	\item $f' < 0 \Rightarrow f$ streng monoton fallend
	\item $f' \leq 0 \Leftrightarrow f$ monoton fallend
\end{itemize}

\subsubsection{Extremstellen}
\begin{itemize}
	\item $f'(x_0) = 0, f''(x_0) > 0 \Rightarrow$ Minimum bei $x_0$
	\item $f'(x_0) = 0, f''(x_0) < 0 \Rightarrow$ Maximum bei $x_0$
	\item $f''(x_0) = 0, f'''(x_0) \neq 0 \Rightarrow$ Wendepunkt in $x_0$
	\item $f'(x_0) = 0, f''(x_0) = 0, f'''(x_0) \neq 0 \Rightarrow$ Sattelpunkt in
  	$x_0$
	\item Extrema bei $x_0 \Rightarrow f'(x_0) = 0$
\end{itemize}

\subsection{Zusammenhang zwischen Stetigkeit und Differenzierbarkeit}
\begin{itemize}
	\item differenzierbar $\Rightarrow$ stetig
	\item differenzierbar $\not\Leftarrow$ stetig
\end{itemize}
