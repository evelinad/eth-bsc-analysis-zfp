\section{Reihen}

\subsection{Definitionen}
Eine Reihe $\sum_{n = 1}^\infty a_n$ ist \underline{konvergent} mit Grenzwert
$s$, wenn die Folge der \underline{Partialsummen} $(S_m)$, $S_m :=
\sum_{n=1}^m a_n$ gegen $s$ konvergiert. Also wenn gilt: $S_m \to s$.

\begin{definition}[$\epsilon$-Kriterium]
	$\forall \epsilon > 0 \; \exists n_0 \in \N \; \forall m \geq n_0: \left|
	\sum_{n=1}^m a_n - s \right| < \epsilon$
\end{definition}

\begin{definition}[Absolute Konvergenz]\index{Konvergenz}
Wenn auch die Reihe der Absolutbeträge $\sum_{n=1}^\infty |a_n|$ konvergiert, so
heisst die Reihe absolut konvergent. Aus der absoluten Konvergenz folgt
Konvergenz. Der Umkehrschluss ist nicht möglich.
\end{definition}

\subsection{Rechenregeln Reihen}
Für \underline{konvergente} Reihen gilt:
\[
	\sum_{n=1}^\infty a_n = A, \sum_{n=1}^\infty b_n = B \Rightarrow
	\sum_{n=1}^\infty (\alpha a_n + \beta b_n) = \alpha A + \beta B
\]

\subsection{Konvergenzkriterien}
\begin{tabular}{|l|}
\hline
	Konvergiert $\sum_{n=1}^\infty a_n$, so ist $\lim_{n \to \infty} a_n = 0$.\\
	Wenn also $\lim_{n \to \infty} a_n \neq 0$, so konvergiert die Reihe
	\underline{nicht}\\
\hline
\end{tabular}

\onecolumn

{\footnotesize
\begin{tabular}{|p{2cm}||p{3cm}|p{3cm}|p{3cm}|p{3cm}|p{3cm}|}
\hline & \multicolumn{4}{l}{schnelles Fallen} & langsames Fallen \\
\hline
 
wie schnell gehen die $a_n$ gegen 0 & exponentiell wie $q^n, |q| < 1$ &
polynominal wie $n^{-\alpha}, \alpha > 1$ & \multicolumn{2}{|c|}{höchstens wie
$1/n$} & gar nicht \\ \hline

Beispiele & \begin{align*} a_n = \frac{n^8}{2^n},\\ a_n =
(\sqrt[n]{n} - 1)^n,\\
a_n = \frac{1}{n!},\\ a_n = \left( \frac{-1}{4}\right)^n \end{align*} &
\begin{align*} a_n = \frac{1}{n^2}, \\ a_n = \frac{1}{n^{100}}, \\ a_n =
\frac{1}{(n + \ln n)^2}, \\ a_n = \frac{20}{n^2 - 33} \end{align*} &
\begin{align*} a_n = \frac{(-1)^n}{\ln n}, \\ a_n = \frac{(-1)^n}{n}
\end{align*} & \begin{align*} a_n = \frac{1}{\ln n}, \\ a_n = \frac{1}{n + \ln
n}, \\ a_n = \frac{1}{n} \end{align*} & \begin{align*} a_n = (-1)^n, \\ a_n =
\sin n, \\ a_n = n^2 \end{align*}\\ \hline

passende Konvergenzkriterien & Wurzel- und Quotientenkriterium & Integral- und
Verdichtungskriterium & Leibniz-Kriterium & & $a_n \not\to 0$ \\ \hline

Vergleichs-, Majoranten-, Minorantenkriterium & Vergleichen mit $q^n$ &
Vergleichen mit $n^{-\alpha}$ & kein Vergleich möglich & Vergleichen mit
$\frac{1}{n}$ & \\ \hline

Konvergenz-verhalten & \multicolumn{2}{c|}{absolute Konvergenz} & keine
absolute Konvergenz (einfach Konvergenz) & \multicolumn{2}{|c|}{Divergenz} \\
\hline

\end{tabular}
}

\begin{tabular}{|p{4cm}|p{15cm}|}\hline
\multicolumn{2}{|c|}{direkte Kriterien}\\ \hline

\textbf{Quotientenkriterium} & Gut für Reihen, die Fakultäten oder Glieder der
Form $a^n$ enthalten. Nicht auf Reihen anwendbar, in denen die Glieder nur wie
eine Potenz von $n$ fallen. \\ \hline

\textbf{Wurzelkriterium} & Gut in Reihen, deren Glieder $n$-te Potenzen sind,
zusammen mit der Stirlingformel oft auch bei Fakultäten anwendbar. \\ \hline

\textbf{Leibnizkriterium} & Nur für alternierende Reihen.\\ \hline

\textbf{Integralkriterium} & Anwendbar auf monotone Reihen.\\ \hline

\multicolumn{2}{|c|}{direkte Kriterien}\\ \hline

\textbf{Vergleichskriterium} & Ermöglicht es "`Störterme"' wegzulassen und so
einfachere Reihen zu untersuchen \\ \hline

\textbf{Verdichtungskriterium} & Bei monotonen Reihen anwendbar. Für Reihen mit
langsam fallenden Gliedern \\ \hline

\textbf{Majoranten- und Minorantenkriterium} & Ähnlich wie Vergleichskriterium.
Wird mit einer Reihe verglichen, deren Glieder stets kleiner oder grösser sind.
\\ \hline
\end{tabular}

\twocolumn

\subsubsection{Reihen Kriterien}
Achtung. Die nachfolgenden Kriterien sagen nur aus, ob die Reihen konvergiert
oder nicht. Sie sagen \underline{nicht} aus, gegen was sie konvergieren!

\paragraph{Quotientenkriterium}
\[
\left| \frac{a_{n+1}}{a_n} \right| \to q. \quad \text{Dann gilt} \begin{cases}
q < 1 & \Rightarrow \sum_{n=1}^\infty a_n \text{ konvergiert absolut} \\
q = 1 & \Rightarrow \text{keine Aussage}\\
q > 1 & \Rightarrow \sum_{n=1}^\infty a_n \text{ divergiert}
\end{cases}
\]

\paragraph{Wurzelkriterium}
\[
\sqrt[n]{\left | a_n \right |} \to q. \quad \text{Dann gilt} \begin{cases}
q < 1 & \Rightarrow \sum_{n=1}^\infty a_n \text{ konvergiert absolut}\\
q = 1 & \Rightarrow \text{keine Aussage}\\
q > 1 & \Rightarrow \sum_{n=1}^\infty a_n \text{ divergiert}
\end{cases}
\]

\paragraph{Leibnizkriterium}
Wenn gilt:
\begin{itemize}
  \item $(a_n)$ ist alternierende Folge, d.h die Vorzeichen wechseln jedes Mal
  \item $a_n \to 0$ oder $|a_n| \to 0$
  \item $(|a_n|)$ ist monoton fallend
\end{itemize}
\ldots dann konvergiert $\sum_{n=1}^\infty a_n$

\paragraph{Majorantenkriterium}
Ist $|a_n| \leq b_n$ und $\sum_{n=1}^\infty b_n$ konvergent, so konvergiert
$\sum_{n=1}^\infty a_n$ absolut.

\paragraph{Minorantenkriterium}
Ist $a_n \geq b_n \geq 0$ und $\sum_{n=1}^\infty b_n$ divergent, so divergiert
$\sum_{n=1}^\infty a_n$
