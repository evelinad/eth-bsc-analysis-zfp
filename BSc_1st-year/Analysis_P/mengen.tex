\begin{multicols}{2}
\section{Mengen}

\subsection{Definitionen}
\begin{description}
	\item [Teilmenge:] $A \subseteq B :\Leftrightarrow \forall x: x \in A \rightarrow x \in B$
	\item [Vereinigung:] $A \cup B := \{x | (x \in A) \lor (x \in B)\}$
	\item [Durchschnitt:] $A \cap B := \{x | (x \in A) \land (x \in B)\}$
	\item [Differenz:] $A \backslash B = A - B := \{x | (x \in A) \land (x \not\in B)\}$
	\item [Komplement:] $A^c = \overline{A} := \{x | x \not\in A\}$
\end{description}

\subsection{Rechenregeln}
{\footnotesize
\begin{tabular}{|l|r|}\hline
$A \cup B = B \cup A$ & $A \cap B = B \cap A$\\
$A \cup (B \cup C) = (A \cup B) \cup C$ & $A \cap (B \cap C) = (A \cap B) \cap C$\\
$A \cup (B \cap C) = (A \cup B) \cap (A \cup C)$ & $A \cap (B \cup C) = (A \cap B) \cup (A \cap C)$\\
$(A \cup B)^c = A^c \cap B^c$ & $(A \cap B)^c = A^c \cup B^c$\\
$(A \backslash B) \cup C = (A \cup C) \cap (B^c \cup C)$ & $(A \backslash B) \cap C = A \backslash )(B \cup C^c)$\\
$(A \backslash B) \backslash C = A \backslash (B \cup C)$ & $A \backslash B = A \cap B^c$\\\hline
\end{tabular}
}

\subsection{Beweise}
Um Mengengleichungen zu beweisen überführt man üblicherweise eine Seite in eine Form,
die nur noch aus logischen Operatoren besteht ($\land, \lor, \in, \not\in$) und formt
dann so um, dass man zur gewünschten anderen Seite kommt durch Rückführung in eine
Form mit Mengenoperatoren. Dazu verwendet man am einfachsten die Definitionen weiter oben.

\subsection{Beispiel}
Zu Zeigen: $(A \cup B)^c = A^c \cap B^c$ wobei $A, B, C$ Untermengen von $X$ sind.
\begin{align*}
(A \cup B)^c &= \{x \in X: x \not\in (A \cup B)\} = \{x \in X: x \not\in A \land x \not\in B\}\\
&= \{x \in X: x \not\in A\} \cap \{x \in X: x \not\in B\} = A^c \cap B^c
\end{align*}

\subsection{bekannte Mengen}
\begin{description}
	\item[$\N$, natürliche Zahlen:] $\{1, 2, 3, \ldots\}$
	\item[$\Z$, ganze Zahlen:] $\{\ldots, -3, -2, -1, 0, 1, 2, 3, \ldots\}$
	\item[$\Q$, rationale Zahlen:] $\{\frac{p}{q} | p \in \Z, q \in \N \backslash \{0\}\}$
	\item[$\R$, reelle Zahlen:] ``alle'' Zahlen, die wir im Alltag brauchen. Genauer: rationale Zahlen und die irrationalen Zahlen.
\end{description}

\subsection{Teilmengen von $\R$}
\subsubsection{Intervalle}
\todo{\ldots}

\subsubsection{Beschränktheit}
Die Menge $M$ sei eine nicht-leere Teilmenge von $\R$ ($M \subset \R, M \neq \emptyset$).

Die Menge $M$ ist \underline{beschränkt}, wenn $C_1, C_2 \in \R$ existieren, sodass gilt: $\forall x \in M: C_1 \leq x \leq C_2$.
Äquivalent dazu ist die Aussage: $\exists C \in \R \; \forall x \in M: |x| \leq C$

Die Menge $M$ ist \underline{nach oben beschränkt}, wenn $C$ existiert, sodass gilt: $\forall x \in M: x \leq C$. \textit{Nach unten beschränkt} ist
entsprechend: $\exists C \in \R \; \forall x \in M: C \leq x$.

\subsubsection{Supremum / Infinum}
Ist $M \subset \R$ nach oben beschränkt, so nennt man jedes $C$ mit $x \leq C, \forall x \in M$
eine \underline{obere Schranke} von $M$. Die kleinste obere Schranke (existiert immer in $\R$) nennt man \underline{Supremum} von $M$ ($\sup M$).

Analog dazu wird die \underline{untere Schranke} und das \underline{Infinum} ($\inf M$) definiert.

Falls die Menge $M$ ein grösstes (bzw. kleinstes) Element besitzt, so nennt man es \underline{Maximum} (bzw. \underline{Minimum}).
Es gilt:
\begin{itemize}
	\item Ist $M \subset \R$ abgeschlossen und beschränkt, so existieren Minimum und Maximum von $M$
	\item Wenn $\max M$ existiert, dann ist $\sup M = \max M$
	\item Ist $\sup M \in M$, so ist $\max M = \sup M$
	\item Wenn $\min M$ existiert, dann ist $\inf M = \min M$
	\item Ist $\inf M \in M$, so ist $\min M = \inf M$
\end{itemize}

\paragraph{mathematische Definition}
$\sup M = a$ gilt genau dann, wenn
\begin{itemize}
	\item $\forall x \in M: x \leq a$, $a$ ist somit obere Schranke von $M$
	\item $\forall \epsilon > 0 \; \exists x \in M: x > a - \epsilon$, d.h. $a - \epsilon$ ist keine obere Schranke mehr, egal wie klein man $\epsilon$ auch wählt $\rightarrow$ $a$ ist kleinste obere Schranke.
\end{itemize}

$\inf M = a$ gilt genau dann, wenn
\begin{itemize}
	\item $\forall x \in M: x \geq a$, $a$ ist somit untere Schranke von $M$
	\item $\forall \epsilon > 0 \; \exists x \in M: x < a + \epsilon$, d.h. $a + \epsilon$ ist keine untere Schranke mehr, egal wie klein man $\epsilon$ auch wählt $\rightarrow$ $a$ ist grösste untere Schranke.
\end{itemize}

\subsection{Mächtigkeit}
Eine Menge $A$ ist gleichmächtig zu einer Menge $B$, wenn es eine \textit{Bijektion}
$f: A \rightarrow B$ gibt. Man schreibt dann $|A| = |B|$.

Hat man zwischen zwei Mengen eine Funktion $f: A \rightarrow B$ gefunden, die bijektiv ist,
so gibt es eine Umkehrfunktion, die ebenfalls Bijektiv ist. Diese bildet jedes Element von $B$
auf eines aus $A$ ab.

\subsubsection{Abzählbar}
Eine Menge $A$ ist abzählbar, wenn sie gleichmächtig zur Menge $\N$ (natürliche Zahlen) ist.

\subsubsection{Gleichmächtigkeit zeigen}
Zeigt man durch angeben einer bijektiven Funktion.

\paragraph{Beispiel}
Zu zeigen: $U := \{ 2k + 1: k \in \N \}$ gleichmächtig zu $\N$ ist.

\textbf{Beweis}: Sei $f: \N \rightarrow U$ gegeben durch
\begin{equation*}
f(n) = \left\{
	\begin{array}{l l}
		n & n \text{ ungerade}\\
		-n - 1 & n \text{ gerade}
	\end{array}
\right.
\end{equation*}

Diese Funktion ist offensichtlich bijektiv (sonst Umkehrfunktion angeben), wodurch $U$ gleichmächtig $\N$ ist.

\subsubsection{weitere gleichmächtige Mengen}
\begin{itemize}
	\item $\N, \Z, \Q$ sind gleichmächtig
	\item $\R, ]0,1[$ sind gleichmächtig
	\item $\R$ ist mächtiger (``überabzählbar'') als $\N$
\end{itemize}

\end{multicols}