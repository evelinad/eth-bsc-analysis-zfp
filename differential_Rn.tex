\section{Differentialrechnung in $\R^n$}
Hier geht es um Funktionen $f: \R^n \to \R^m$, wobei $m=1$ gelten kann
($f: \R^n \to \R$). Solche Funktionen haben die allgemeine Form:
$f(x) = f(x_1, x_2, x_3, \ldots, x_n) = \begin{pmatrix}
f_1(x_1, x_2, x_3, \ldots, x_n)\\
f_2(x_1, x_2, x_3, \ldots, x_n)\\
\ldots\\
f_m(x_1, x_2, x_3, \ldots, x_n)
\end{pmatrix}$

Für nahezu alle Eigenschaften gilt: Die Vektorfunktion $f: \R^n \to \R^m$ hat
eine bestimmte Eigenschaft, wenn jede einzelne ihrer Komponenten
($f_1, f_2, \ldots, f_m$) die besagte Eigenschaft besitzen. Das Problem liegt
neu also nicht im Wertebereich, sondern vor allem in Definitionsbereich.

\subsection{Norm}
Eine Norm auf $\R^n$ ist die Funktion $\|\cdot\|: \R^n \to \R$ mit den folgenden
Eigenschaften:
\begin{itemize}
	\item $\forall x \in \R^n: \|x\| \geq 0$
	\item $\forall x \in \R^n: \|x\| = 0 \Leftrightarrow x = \vec{0}$
	\item $\forall x \in \R^n, \alpha \in \R: \|\alpha x\| = |\alpha| \|x\|$
	\item $\forall x,y \in \R^n: \|x + y\| \leq \|x\|+\|y\|$
\end{itemize}

\subsection{Partielle Differenzierbarkeit}
$f: \R^n \to \R^m$ ist in $a = (a_1, \ldots, a_n)$ partiell differenzierbar nach
der $i$-ten Variable $x_i$, wenn die Funktion
$f: x_i \to f(x_1, \ldots, x_i, \ldots, x_n)$ differnzierbar ist. Man berechnet
die partielle Ableitung also folgendermassen: Eine Funktion $f$ wird nach einer
Variable partiell differenziert, indem man alle anderen Variablen als Konstanten
behandelt und die Rechenregeln für Funktionen mit einer Variable anwendet.

\begin{satz}[Satz von Schwarz]
Ist $f$ nach $x$ und $y$ zweimal partiell differenzierbar und sind die gemischten
partiellen Ableitungen $f_{xy}$ und $f_{yx}$ stetig, so gilt: $f_{xy} = f_{yx}$.
\end{satz}

\subsection{Vektoranalysis}
\begin{definition}[Vektorfeld]
Die Abbildung $\vec{v}(\vec{r}): \R^n \to \R^n$ ist ein Vektorfeld. Es weist jedem
Vektor $\vec{r}$ einen Vektor $\vec{v}(\vec{r})$ zu.
\end{definition}

\begin{definition}[Skalarfeld]
Ist eine Abbildung der Form $f: \R^n \to \R$.
\end{definition}

\begin{definition}[Gradient]
Ist $f: \R^n \to \R$ (Skalarfeld), so ist der Gradient von $f$ der Vektor
$\vec{v} = \grad f = \nabla f$. (Im Fall $f: \R^3 \to \R$:   $\grad f = (f_x, f_y, f_z) =
\left( \frac{\partial f}{\partial x}, \frac{\partial f}{\partial y}, \frac{\partial f}{\partial z} \right)$).
\end{definition}

\begin{definition}[Potential]
Ist $\vec{v} = \grad f$ der Gradient von $f$, so ist $f$ das Potential oder Stammfunktion zu $\vec{v}$.
\end{definition}

\begin{definition}[Gradientenfeld / Potentialfeld]
Ist $\vec{v} = \grad f$ der Gradient von $f$, so ist das Vektorfeld $\vec{v}$
ein Gradientenfeld / Potentialfeld. Es besitzt dabei die folgenden
Eigenschaften:
\begin{itemize}
	\item Der Wert des Kurvenintegrals entlang eines beliebigen Weges innerhalb des
	Feldes ist unabhängig vom Weg selbst, sondern nur vom Anfangs- und Endpunkt
	\item Ein Kurvenintegral mit einem Weg bei dem Anfangs- und Endpunkt der
	gleiche Punkt sind, hat den Wert 0.
	\item Ist immer wirbelfrei: $\rot \vec{v} = \rot(\grad f) = \vec{0}$
\end{itemize}
\end{definition}

\begin{definition}[Rotor / Rotation]
Ist $\vec{v}$ ein Vektorfeld im $\R^3$, so ist die Rotation von $\vec{v} = (P, Q, R)^T$ das Vektorfeld
$\vec{w} = \rot \vec{v} = \begin{pmatrix}
R_y - Q_z\\
P_z - R_x\\
Q_x - P_y
\end{pmatrix}$
\end{definition}

\begin{definition}[Vektorpotential]
Ein Vektorfeld $\vec{v}$ heisst Vektorpotential zu $\vec{w}$, falls $\vec{w} = \rot \vec{v}$.
\end{definition}

\begin{definition}[Wirbelfrei / konservativ]
Ist $\vec{v}$ ein Vektorfeld mit $\rot \vec{v} = 0$, so nennt man $\vec{v}$ wirbelfrei
\end{definition}

\begin{definition}[Divergenz]
Die Divergenz eines Vektorfelds $\operatorname{div} \vec{K}(x,y,z)$ ist definiert durch:
$\operatorname{div} \vec{K}(x,y,z) := \left( \frac{\partial
K_1}{\partial x} + \frac{\partial K_2}{\partial y} + \frac{\partial
K_3}{\partial z} \right)$
\end{definition}

\subsubsection{Bestimmung eines Potentials im $\R^2$}
Sei $\vec{v} = \begin{pmatrix}
P(x,y)\\
Q(x,y)
\end{pmatrix}$.

Um schnell zu prüfen, ob man überhaupt den folgenden Algorithmus anwenden muss,
kann man prüfen ob gilt: $P_y = Q_x$, wenn nicht, so hat $\vec{v}$ kein Potential $f$.

\begin{enumerate}[itemsep=1em]
	\item $f(x,y) = \int P(x,y)\;dx + C(y)$ berechnen (Integral berechnen)
	\item Die berechnete Gleichung $f(x,y)$ nun nach $y$ ableiten:
	$\frac{\partial}{\partial y} f(x,y) = \frac{\partial}{\partial y}\int P(x,y)\;dx + C'(y)$
	(berechnetes Integral nach $y$ ableiten)
	\item $\frac{\partial}{\partial y} f(x,y) = Q(x,y)$ setzen und $C'(y)$ berechnen durch umformen
	und integrieren
	\item Berechnetes $C(y)$ in die Gleichung im 1. Punkt einsetzen. Fertig. Achtung: Im Grunde hat
	$C(y)$ durch integrieren (aufleiten) noch einen konstanten Wert, der beliebigen Wert haben kann.
	Dieser taucht im Grunde auch in der fertigen $f(x,y)$ Funktion auf.
\end{enumerate}

\subsubsection{Bestimmung eines Potentials im $\R^3$}
Sei $\vec{v} = \begin{pmatrix}
P(x,y,z)\\
Q(x,y,z)\\
R(x,y,z)
\end{pmatrix}$.

Um zu prüfen, ob man überhaupt ein Potential finden kann für $\vec{v}$ hat $\rot \vec{v} = 0$
zu sein, also wirbelfrei zu sein. Dazu muss gelten (zu zeigen mit $\rot \vec{v} = 0$): $P_y = Q_x, P_z = R_x, Q_z = R_y$.

\begin{enumerate}[itemsep=1em]
	\item $f(x,y,z) = \int P(x,y,z)\;dx + C(y,z)$ lösen (Integral berechnen)
	\item Nun die berechnete Gleichung $f(x,y,z)$ nach $y$ ableiten $\Rightarrow f_y(x,y,z)$.
	\item Die abgeleitete Gleichung $f_y$ mit $Q(x,y,z)$ gleichsetzen: $f_y(x,y,z) = Q(x,y,z)$
	und damit $C_y(y,z)$ bestimmen.
	\item Durch Integration von $C_y(y,z)$ nach $y$ ($\int C_y(y,z)\;dy$) wird $C(y,z)$ bestimmt
	bis auf eine Konstante $D(z)$, die von $z$ abhängt. $C(y,z)$ hat also die Form:
	$C(y,z) = \int C_y(y,z)\; dy + D(z)$.
	\item Dieses $C(y,z)$ setzt man nun in die Gleichung $f(x,y,z)$ ein, die im 1. Punkt steht.
	\item Nun wird die daraus erzeugte
	$f(x,y,z) = \int P(x,y,z)\;dx + C(y,z) = \int P(x,y,z)\;dx + \int C_y(y,z)\; dy + D(z)$
	Gleichung nach $z$ abgeleitet.
	\item Durch Gleichsetzen von $f_z(x,y,z) = R(x,y,z)$ lässt sich $D_z(z)$ bestimmen.
	\item $D_z(z)$ wird wiederrum durch Integration zu $D(z) = \int D_z(z)\; dz + c, \quad c \in \R$
	\item Das berechnete $D(z)$ in die $f(x,y,z)$ Gleichung aus Punkt 6 einsetzen, fertig.
\end{enumerate}