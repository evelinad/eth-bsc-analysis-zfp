\section{Kurvenintegral (Linienintegral)}
In den Übungen sind nur immer Integrale der zweiten Art vorgekommen. 

\subsection{2. Art}
Das Wegintegral über ein \underline{stetiges Vektorfeld} $\vec{f}: \R^n \to \R^n$
entlang eines stetig differenzierbaren Weges $\gamma: [a,b] \to \R^n$ ist definiert
durch:
\[
\int_\gamma \vec{f}(\vec{x}) d\vec{x} := \int_a^b \left< \vec{f}(\gamma(t)), \gamma(t)' \right> dt
\]

\underline{Skalarprodukt}: $\left< \vec{a}, \vec{b} \right> = a_x b_x + a_y b_y + \ldots$

\subsubsection*{Beispiel}
%Beispiel Analysis II Serie 11 4a)
Berechne des Linienintegral $\int_\gamma \vec{K} d\vec{x} $ für $ \vec{K}(x,y) = (x^2+y, 2xy)$ und $\gamma$ als Einheitskreis mit positivem Umlaufsinn.
Gegeben:
\begin{align*}
\vec{K}(x,y) &= (x^2+y, 2xy) \\
\gamma &: [0,2\pi]  \to \R^2\\
\gamma &: t \mapsto   (\cos(t), \sin(t)) 
\end{align*}
Zu berechnen:
\begin{eqnarray*}
\gamma(t)' &=& (-\sin(t),\cos(t))\\
\vec{K}(\gamma(t)) &=& (\cos^2(t) + \sin(t), 2 \cos(t)\sin(t))\\
\left< \vec{K}(\gamma(t)), \gamma(t)' \right> &=& -\cos^2(t)\sin(t) - \sin^2(t)
+ 2\cos^2(t)\sin(t)\\
 &=& \cos^2(t)\sin(t) - \sin^2(t)
\end{eqnarray*}
\begin{eqnarray*}
\int_\gamma \vec{K} d\vec{x} &=& \int_0^{2\pi} \left< \vec{K}(\gamma(t)), \gamma(t)' \right> dt\\
&=& \int_0^{2\pi} \cos^2(t)\sin(t) - \sin^2(t) dt\\
&=& \left. -\frac{1}{3}\cos^3(t) - \frac{1}{2}t + \sin(t)\cos(t)
\right|_0^{2\pi} = \underline{\underline{-\pi}}
\end{eqnarray*}


\subsection{1. Art}
Das Wegintegral einer \underline{stetigen Funktion} $f: \R^n \to \R$ entlang
eines stetig differenzierbaren Weges $\gamma: [a,b] \to \R^n$ ist definiert durch:
\[
\int_\gamma f ds := \int_a^b f(\gamma(t)) \|\gamma(t)'\|_2 dt
\]

\underline{Euklidische Norm}: $\|\vec{a}\|_2 = \sqrt{a_x^2 + a_y^2 + \ldots}$.
Achtung: Beim Integral muss man zuerst $\gamma(t)$ nach $t$ ableiten und erst
dann die Norm davon berechnen!

\subsubsection*{Beispiel}
% Beispiel von http://www.tu-ilmenau.de/fileadmin/media/num/neundorf/Dokumente/Lehre/hm/Kurven_Integral.pdf Seite 2
Es sei die Schraubenlinie (Spirale in 3D)
\[
\gamma : [0,2\pi]  \to \R^3, \gamma : t \mapsto  (\cos(t), \sin(t), t) 
\]
und $f(x,y,z) := x^2 + y^2 + z^2$ gegeben. Wir berechnen $\int_\gamma f ds$. Zunächst bestimmen wir
\begin{align*}
\|\gamma(t)'\|_2 &= \sqrt{\left[\frac{d(\cos t)}{dt}\right]^2 +
\left[\frac{d(\sin t)}{dt}\right]^2 + \left[\frac{dt}{dt}\right]^2} \\
&= \sqrt{\sin^2(t)+\cos^2(t)+1}=\sqrt{2}
\end{align*}
Dann substituieren wir x,y und z und erhalten
\[
f(x,y,z) = f(\gamma(t)) = \sin^2(t)+\cos^2(t)+t^2 = 1 +t^2
\]
auf $\gamma$. Das führt zu
\begin{align*}
\int_\gamma f(x,y,z) ds &= \int_0^{2\pi} (1 +t^2)\sqrt{2} dt = \left. \sqrt{2}(t+\frac{t^3}{3}) \right|_0^{2\pi} \\
&= \frac{2\sqrt{2}\pi}{3}(3+4\pi^2)
\end{align*}


\subsection{Parametrisierung von Kurven}
Grundlegender Tipp: Skizze machen, um Grenzen und Kurve besser zu verstehen und
schneller auf die Parametrisierung zu kommen.

\begin{itemize}[leftmargin=*]
	\item Wenn die Kurve in der Form
	\[
	C = \{\vec{r} \in \R^n | \vec{r} = \gamma(t), a \leq t \leq b\}
	\]
	bereits gegeben, so ist klar, dass $\gamma(t)$ der Weg ist und das Integral von
	$a$ nach $b$ verläuft.
	
	\item Die Paramtrisierung einer \underline{Strecke} von $\vec{a}$ nach $\vec{b}$:
	$\gamma(t) = \vec{a} + t(\vec{b}-\vec{a}), \quad 0 \leq t \leq 1$
	
	\item Die Parametrisierung eines \underline{Kreises} mit Mittelpunkt $(x_0, y_0)$ und
	Radius $r$ ist: $\gamma(t) =
	\begin{pmatrix}
	x_0 + r \cos(t)\\
	y_0 + r \sin(t)
	\end{pmatrix}$. Für einen vollen Kreis gilt $0 \leq t \leq 2\pi$, für Kreisteile
	schränkt man diesen Intervall entsprechend ein.
	
	\item Parametrisierung eines \underline{Graphen} der Funktion $f(x)$ für $x$
	zwischen $a$ und $b$: $\gamma(t) =
	\begin{pmatrix}
	t\\
	f(t)
	\end{pmatrix}, \quad a \leq t \leq b$. \underline{Achtung:} Hat man zwei
	Graphen, die als Grenzen für das Kurvenintegral fungieren, so schliessen diese
	gemeinsam eine Fläche ein. Die Umlaufrichtung ist so zu wählen, dass diese
	Fläche jeweils links liegt.
\end{itemize}

\subsubsection{Beispiel: Kurvenintegral mit zwei Grenzgraphen}
% Analysis 2, Serie 12, Aufgabe 2b
Es soll das Kurvenintegral $\int_\gamma K(x,y)\;dt = \int_\gamma
\begin{pmatrix}
x^2 - y^2\\
2y -x
\end{pmatrix}\;dt$ berechnet werden. $\gamma$ ist der Rand des beschränkten
Gebietes im ersten Quadranten, welches durch die Graphen $y=x^2$ und $y=x^3$ begrenzt
wird.

Lösung: Man macht eine Skizze der Grenzen ($y=x^2$ und $y=x^3$) und sieht, dass
diese eine Fläche einschliessen. Die Schnittpunkte sind $x = 0$ und $x = 1$. Der
Rand dieses Gebiets besteht also aus den beiden parametrisierten Kurven
(Parametrisierung für Graphen verwenden)
\begin{align*}
\gamma_1(t) &= \begin{pmatrix}t\\t^3\end{pmatrix} & t \in [0,1]\\
\gamma_2(t) &= \begin{pmatrix}t\\t^2\end{pmatrix} & t \in [0,1]
\end{align*}

Wir sehen, dass die eingeschlossene Fläche in Durchlaufrichtung von $\gamma_1$
links liegt, was soweit gut ist. Bei $\gamma_2$ liegt die Fläche aber auf der
rechten Seite, weshalb wir die Durchlaufrichtung drehen müssen, was zu $\gamma
= \gamma_1 - \gamma_2$ führt (man beachte das Minus statt einem Plus).

Jetzt muss nur noch ganz normal das Wegintegral berechnet werden: $\int_\gamma
K\;dx = \int_{\gamma_1} K\;dx + \int_{-\gamma_2} K\;dx = \int_{\gamma_1} K\;dx -
\int_{\gamma_2} K\;dx = \ldots$

\subsection{Berechnung}
Die Berechnung findet in drei Schritten statt:
\begin{enumerate}[leftmargin=*]
	\item Parametrisierung der Kurve $C$ als $C = \{\vec{r} \in \R^n | \vec{r} = \gamma(t), a \leq t \leq b\}$
	\item Einsetzen ins Integral: $\int_C f(\vec{r})\,ds = \int_a^b f(\gamma(t)) \|\gamma(t)'\|\,dt$.
	Man setzt also für die Variabeln von $f$ die Komponenten von $\gamma$ ein und
	multipliziert dies dann mit dem Betrag der Ableitung nach $t$ von $\gamma$.
	\item Integral ausrechnen.
\end{enumerate}

\subsection{Bogenlänge}
Für die Bogenlänge L, auf dem Weg beschrieben durch $x(t)$ und $y(t)$, ergibt sich: \[
L = \int ds = \int_a^b \sqrt{\dot{x}^2 + \dot{y}^2} dt
\]

