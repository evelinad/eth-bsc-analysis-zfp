\section{Stetigkeit von Funktinen mit mehreren Variabeln}
\begin{definition}[Grenzwert]
Eine Funktion $f(\vec{x})$ hat einen \underline{Grenzwert} $b$ an der Stelle
$\vec{a}$, wenn für \underline{jede} vektorwertige Folge $(\vec{x}_k)_{k \in \N}$,
mit $\vec{x}_k \neq \vec{a}$, $\lim_{k \to \infty}f(\vec{x}_k) = b$ erfüllt ist
\end{definition}
\begin{definition}[Stetigkeit]
Die Funktion $f(\vec{x})$ ist stetig im Punkt $\vec{a}$, wenn für
\underline{beliebige} Folgen $(\vec{x}_k)_{k \in \N}$ mit
$(\vec{x}_k) \to \vec{a}$ gilt: $\lim_{\vec{x}_k \to \vec{a}} f(\vec{x}_k) = f(\vec{a})$
für alle $k$ gilt.
\end{definition}

\subsection{Beispiel 1}
\[
f(x, y) = \begin{cases}
	\frac{4xy}{x^2 + y^2} & (x,y) \neq (0,0)\\
	0 & (x,y) = (0,0)
\end{cases}
\]

Die Funktion ist als Komposition, Addidition und Multiplikation stetiger
Funktionen überall ausser in $(0,0)$ stetig (1. Zeile). Es bleibt zu untersuchen,
ob $f$ auch in $(0,0)$ stetig ist.

Längs der $x$-Achse ($y = 0$) erhalten wir:
\begin{align*}
f(x, 0) = \frac{4x \cdot 0}{x^2 + 0^2} = 0\\
\underline{\lim_{x \to 0} f(x, 0)} = \lim_{x \to 0} 0 = \underline{0}
\end{align*}

Als nächstes untersuchen wir die Winkelhalbierende ($x = y$):
\begin{align*}
f(x, x) = \frac{4x^2}{x^2 + x^2} = \frac{4x^2}{2x^2} = 2\\
\underline{\lim_{x \to 0} f(x, x)} = \lim_{x \to 0} 2 = \underline{2}
\end{align*}

Die Funktion ist somit in $(0,0)$ nicht stetig, da
$\lim_{x \to 0} f(x, 0) \neq \lim_{x \to 0} f(x, x)$ ist.