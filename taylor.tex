\section{Taylorreihe / -entwicklung}
Funktionen werden in der Umgebung eines bestimmten Punktes durch eine
Potenzreihe dargestellt. Damit wird die ursprüngliche Funktion angenähert.

Die Taylorreihe der Funktion $f$ mit Grad $m$ um den Punkt $a$ ist:
$T_m f(x;a) = \sum_{n = 0}^m \frac{f^{(n)}(a)}{n!}(x - a)^n$\newline
{\small
$= f(a) + \frac{f'(a)}{1!}(x-a) + \frac{f''(a)}{2!}(x-a)^2 +
\frac{f^{(3)}(a)}{3!}(x-a)^3 + \frac{f^{(4)}(a)}{4!}(x-a)^4\ldots$
}

Dabei ist $n! = 1 \cdot 2 \cdot 3 \cdot \ldots \cdot n$

\subsection{Rechenregeln}
\subsubsection{Addition}
$f, g$ sind $m$-mal differenzierbar:

$T_m (f + g)(x;a) = T_m f(x;a) + T_m g(x;a)$

\subsubsection{Multiplikation}
$f, g$ sind $m$-mal differenzierbar:

$T_m (f \cdot g)(x;a) = T_m(T_m f(x;a) \cdot T_m g(x;a))$

\underline{Achtung}: Anschaulich bedeutet es folgendes: Man
multipliziert die beiden Taylorreihen von $f$ und $g$ miteinander ($T_m f(x;a) \cdot T_m g(x;a)$).
Danach entfernt man alle Terme der Ordnung $> m$.

\subsubsection{Kettenregel}
$f: A \to B, g: B \to \R$ zwei $m$-mal differenzierbare Funktionen.
Entwickelt wird um den Punkt $a \in A$ mit $f(a) = q$ ($q$ muss man berechnen).
Dann gilt:
\begin{align*}
T_m (g \circ f)(x;a) &= T_m (f(g))(x;a)\\
&= T_m(T_m(g)(x;q) \circ (T_m(f)(x;a) - q))
\end{align*}

\subsubsection{Bemerkungen / Eigenschaften / Konvergenz}
\begin{itemize}
	\item Der Konvergenzradius kann 0 sein
	\item Falls Taylor-Reihe konvergiert, dann ist sie nicht notwendig gleich
	der Funktion, die sie beschreibt. Gegenbeispiel:
	$f(x) = \begin{cases}
	e^{-\frac{1}{x}} & x > 0\\
	0 & x \leq 0\end{cases}$
	\item Ist $f$ eine Potenzreihe, dann ist diese Potenzreihe auch die Taylor-Reihe
\end{itemize}