\section{Differentialgleichung (DGL)}
\subsection{Lineare DGL 1. Ordnung ($y' + f(x) \cdot y = g(x)$)}
\subsubsection{Variation der Konstanten}
%Nach http://de.wikipedia.org/wiki/Variation_der_Konstanten#Motivation
Sei:\[
F(x) = \int f(t) dt.
\]
Dann ist $\{y_{Hom}(x) = c_1e^{F(x)}| c_1 \in \R\}$ die Menge aller Lösungen der homogenen Differnentialgleichung ($y' + f(x) \cdot y = 0$). Als Ansatz für die Lösung des inhomogenen Problems setze man $y_p(x) = u(x)e^{F(x)}$, d.h. man \textit{lässt die Konstante $c_1$ variieren}. Dies ergibt eine eindeutige Zuordnung zwischen den Funktion $y$ und $u$. Denn $e^{F(x)}$ ist eine stets positive, stetig differnezierbare Funktion. Die Ableitung dieser Ansatzfunktion ist \[
y_p'(x) = u(x)f(x)e^{F(x)} + u'(x)e^{F(x)} = y(x)f(x) + u'(x)e^{F(x)}
\]
Also löst $y$ die inhomogene Differntialgleichung \[
y_p'(x) = y(x)f(x) + g(x)
\]
genau dann, wenn\[
u'(x) = y(x)f(x) + g(x)
\]
gilt. Also folgt\[
u(x) = \int  g(t)e^{-F(t)} dt
\]
Somit ist die Lösungsmenge von $y_p$
\[
\{ y_p'(x) = e^{F(x)} (u(x) + c_2) | c_2 \in \R \}
\]
Die Lösungsmenge der generellen Lösung der allgemeinen Lösung ist somit $y$:
\begin{align*}
&\{y\}=\{y = y_{Hom} + y_p\}\\
& =\{ y(x) =  c_1e^{F(x)} +  e^{F(x)} (u(x) + c_2) | c_1, c_2 \in \R\} 
\end{align*}
Gibt es nun einen Ansatz, kann diese menge durch Einsetzen des Funktionswert und Gleichsetzen mit dem Resultat genau bestummen werde in dem die Konstanten $c$  aufgelöst werden.\\
\\
Konkret reicht es also aus. $F$ und $u$ zu berechnen. 

\subsubsection{genereller Ansatz}
Wenn $g(x) = 0$ ist, dann ist die DGL homogen. Falls $g(x) \neq 0$, so handelt
es sich um eine inhomogene DGL.

Der erste Schritt für homogene und inhomogene DGL ist die Lösung der homogenen
DGL: $y' + f(x) \cdot y = 0$:
{\small
\begin{align*}
y' + f(x) \cdot y &= 0 \quad \left | -(f(x) \cdot y) \right.\\
y' &= -f(x) \cdot y \quad \boxed{y' \text{ ist das gleiche wie } \frac{dy}{dx}}\\
\frac{dy}{dx} &= -f(x) \cdot y \quad \left | \div y \right.\\
\frac{dy}{dx\, y} &= -f(x) \quad \left | \int \right.\\
\int \frac{dy}{dx\, y} dx &= \int -f(x) dx \quad \boxed{\frac{dy}{dx\, y} \cdot dx = \frac{dx}{dx\, y} dy = \frac{1}{y} dy}\\
\int \frac{1}{y} dy &= \int -f(x) dx\\
\ln(y) &= -F(x) \quad \left | e^\alpha \right.\\
e^{\ln(y)} &= e^{-F(x)}\\
y &= e^{-F(x)}
\end{align*}
}

Damit erhalten wir die allgemeine Lösung: $y = A \cdot e^{-F(x)}$. Hat man eine
homogene DGL und einen Punkt, an dem die ursprüngliche Funktion ausgewertet wurde,
so kann man die explizite Lösung berechnen (also $A$ berechnen), in dem man die
hier allgemein erhaltene Lösung für den gegebenen Punkt auswertet und so die
Unbekannte bekommt.

Für ein inhomogenes DGL setzt sich die allgemeine Lösung aus der homogenen Lösung
$y_h$ und der partikulären (speziellen) Lösung $y_p$ der inhomogenen DGL zusammen.
Die homogene Lösung haben wir bereits berechnet: $y_h = A \cdot e^{-F(x)}$. Nun
folgt die partikuläre Lösung:

Dazu wird die Konstante ($A$) der homogenen Lösung als Funktion dargestellt ($u(x)$).
Wir erhalten somit: $y_p = u(x) \cdot e^{-F(x)}$.
Dieses $y_p$ setzten wir nun als $y$ in die inhomogene Gleichung ein:
{\small
\[
y' + f(x) \cdot y = g(x) \Rightarrow (\underbrace{u(x) \cdot e^{-F(x)}}_{= y_p = y})'
+ f(x) \cdot (\underbrace{u(x) \cdot e^{-F(x)}}_{= y_p = y}) = g(x)
\]
}

Die neue Gleichung wird nun nach $u'(x) = \ldots$ aufgelöst, was zu
$u'(x) = \frac{g(x)}{e^{-F(x)}}$ führt. Nun wird $u(x)$ bestimmt durch integrieren
beider Seiten: $u(x) = \int \frac{g(x)}{e^{-F(x)}}\,dx$. Hat man dies ausgerechnet,
setzt man $u(x)$ in $y_p = u(x) \cdot e^{-F(x)}$ ein und bekommt so die partikuläre
Lösung der DGL.

Als letzter Schritt für inhomogene DGL summiert man $y_h$ und $y_p$ und erhält nach
dem Umformen und Kürzen die allgemeine Lösung der DGL:
{\small
\[
y = y_h + y_p = 
\underbrace{A \cdot e^{-F(x)}}_{= y_h} +
\underbrace{\underbrace{\int \frac{g(x)}{e^{-F(x)}}\,dx}_{= u(x)} \cdot e^{-F(x)}}_{= y_p}
\]
}

Hat man für die inhomogene DGL ebenfalls Punkte an denen die Funktion ausgewertet wurde,
so kann man dies in die allgemeine Lösung eintragen und so die Unbekannten ($A$) berechnen.

\subsubsection{Beispiel mit Variation der Konstanten}
%Nach http://de.wikipedia.org/wiki/Variation_der_Konstanten#Motivation
Gegeben: $y' + x^2 \cdot y = 2x^2$\\
Somit:  $y' = f(x) \cdot y + g(x)$ mit $g(x) := 2x^2$ und $f(x) :=  - x^2$\\
Es gilt somit:\[
F(x) = \int f(t) dt. =  \int - x^2 dt. = -\frac{x^3}{3}
\]
Dann ist \[
\{y_{Hom}(x) = c_1e^{-\frac{x^3}{3}}| c_1 \in \R\}
\] die Menge aller Lösungen der homogenen Differnentialgleichung ($y' + x^2\cdot y = 0$). \\

\fbox{%
        \parbox{1\linewidth}{%
\textit{Dieser Teil muss nicht berechnet werden (Herleitung).}\\
Als Ansatz für die Lösung des inhomogenen Problems setze man $y_p(x) = u(x)e^{F(x)}$.
\\
Ansatzfunktion ist \[
y_p'(x) = u(x)f(x)e^{F(x)} + u'(x)e^{F(x)} = y(x)f(x) + u'(x)e^{F(x)}
\]
Also löst $y$ die inhomogene Differntialgleichung \[
y_p'(x) = y(x)f(x) + g(x)
\]
genau dann, wenn\[
u'(x) = y(x)f(x) + g(x)
\]
gilt.
        }%
}

Also folgt\[
u(x) = \int  g(x)e^{-F(x)} dx. =  2 \int  x^2e^{\frac{x^3}{3}} dx. = 2e^{\frac{x^3}{3}}
\]
Somit ist die Lösungsmenge von $y_p$
\[
\{ y_p'(x) = e^{-\frac{x^3}{3}} (2e^{\frac{x^3}{3}} + c_2) | c_2 \in \R \}
= \{ y_p'(x) =2 + c_3) | c_3 \in \R \}
\]
Die Lösungsmenge der generellen Lösung der allgemeinen Lösung ist somit $y$\[
\{ y(x) =  c_1e^{-\frac{x^3}{3}} +  2  | c_1 \in \R \}
\]

\subsubsection{Beispiel genereller Ansatz}
Gegeben: $y' + x^2 \cdot y = 2x^2$

Homogene DGL lösen: $y' + x^2 \cdot y = 0$
\begin{align*}
y' + x^2 \cdot y &= 0\\
\frac{dy}{dx} + x^2 \cdot y &= 0 \quad | -(x^2 \cdot y)\\
\frac{1}{dx}\, dy &= -x^2 \cdot y \quad | \div y\\
\frac{1}{dx} \frac{1}{y} \, dy &= -x^2 \quad | \int\\
\int \frac{1}{dx} \frac{1}{y} \, dy \, dx &= \int -x^2 \, dx\\
\int \frac{1}{y}\, dy &= \int -x^2 \, dx\\
\ln(y) &= -\frac{1}{3} x^3 \quad | e^\alpha\\
y &= e^{-\frac{1}{3}x^3}
\end{align*}

Somit ist die allgemeine homogene Lösung: $\underline{y_h = A \cdot e^{-\frac{1}{3}x^3}}$.


Als nächstes gehen wir die praktikuläre Lösung an:
$y_p = u(x) \cdot e^{-\frac{1}{3}x^3}$
\begin{align*}
\Rightarrow (u(x) \cdot e^{-\frac{1}{3}x^3})' + x^2 (u(x) e^{-\frac{1}{3}x^3}) &= 2 x^2\\
u'(x) \cdot e^{-\frac{1}{3}x^3} - u(x) \cdot x^2 e^{-\frac{1}{3}x^3} + u(x) x^2 e^{-\frac{1}{3}x^3} &= 2 x^2\\
u'(x) \cdot e^{-\frac{1}{3}x^3} &= 2 x^2 \quad | \div e^{-\frac{1}{3}x^3}\\
u'(x) &= 2 x^2 e^{\frac{1}{3}x^3} \quad | \int\\
u(x) &= 2 e^{\frac{1}{3}x^3}
\end{align*}

Wir erhalten somit: $\underline{y_p} = 2 e^{\frac{1}{3}x^3} \cdot e^{-\frac{1}{3}x^3} = \underline{2}$.
Die allgemeine Lösung des inhomogenen DGL ist somit:
$\underline{\underline{y}} = y_h + y_p = \underline{\underline{A \cdot e^{-\frac{1}{3}x^3} + 2}}$


\subsubsection{Beispiel Direkterer Lösungsweg}
Gegeben: $y' + x^2 \cdot y = 2x^2$.
Direkt lösen:
\begin{equation*}
\begin{array}{r l |l}
y' + x^2 \cdot y &= 2x^2\\
\frac{dy}{dx} + x^2 \cdot y &= 2x^2 \quad & -(x^2 \cdot y)\\
\frac{dy}{dx} &= 2x^2 -x^2 \cdot y \quad & \text{vereinfachen}\\
\frac{dy}{dx} &= x^2 ( 2 - y) \quad & \div (2 - y) \\
\frac{\frac{dy}{dx}}{2 - y} &= x^2  & \int \text{ with respect to } x \\
\int \frac{\frac{dy}{dx}}{2 - y}  \, dx &= \int x^2 \, dx & \text{links: } \int \frac{g'(x)}{g(x)} \, dx = \ln|g(x)|\\ 
- ln |2 - y| &= \frac{x^3}{3} + c_1 & \cdot (-1) \\
ln |2 - y| &= - \frac{x^3}{3} - c_1 & e \\
e^{ln |2 - y|} &= e^{-\frac{1}{3}x^3 - c_1 } \\
2 - y &= e^{-\frac{1}{3}x^3 - c_1 }& - 2 \\
- y &= e^{-\frac{1}{3}x^3 - c_1 } - 2 & \cdot (-1)\\
y &= - e^{-\frac{1}{3}x^3 - c_1 } + 2 & \text{replace const} \\
\underline{\underline{y}} &= \underline{\underline{c_2 \cdot e^{-\frac{1}{3}x^3} + 2}}\\
\end{array} 
\end{equation*}


\subsection{Lineare DGL höherer Ordnung}
Hier geht es um DGL der Form:
$a_n y^{(n)}+a_{(n-1)}y^{(n-1)}+\ldots+a_1 y'+a_0=g(x)$.

Wieder unterscheiden wir homogene DGL ($g(x) = 0$) und inhomogene DGL ($g(x) \neq 0$).

Als ersten Schritt lösen wir für homogene und inhomogene DGL die homogene Version der DGL:
$a_n y^{(n)}+a_{(n-1)}y^{(n-1)}+\ldots+a_1 y'+a_0 y = 0$. Dazu ersetzen wir $y^{(n)}$ durch
$\lambda^n \cdot e^{\lambda x}$. Beispielsweise wird aus $a_2 y'' + a_1 y' + a_0 y = 0$ wird
$a_2 \lambda^2 e^{\lambda x} + a_1 \lambda e^{\lambda x} + a_0 e^{\lambda x} = 0$.

Diese neue Gleichung ist das charakteristische Polynom. Von diesem berechnen wir
als erstes die Nullstellen $\lambda_i$. Wir beachten, dass wir auch komplexe Nullstellen
miteinbeziehen. Auch merken wir uns die Vielfachheit einer Nullstelle.

Hat man die Nullstellen $\lambda_i$, so lösen für einfache Nullstellen $e^{\lambda_i x}$
die homogene DGL. Ist die Nullstelle $\lambda_i$ $k$-fach, so lösen
$e^{\lambda_i x}, x e^{\lambda_i x}, x^2 e^{\lambda_i x}, \ldots, x^{k-1} e^{\lambda_i x}$
die homogene DGL. Wir erhalten also die allgemeine homogene Lösung in einer ähnlichen Form wie:
$y_h = C_n e^{\lambda_n x} + 
\underbrace{C_{n-1} e^{\lambda_{n-1} x} + C_{n-2} x e^{\lambda_{n-1} x}}_{\lambda_{n-1} \text{: 2-fache Nullstelle}} + \ldots + 
\underbrace{C_3 x^2 e^{\lambda_1 x} + C_2 x e^{\lambda_1 x} + C_1 e^{\lambda_1 x}}_{\lambda_1 \text{: 3-fache Nullstelle}}$.

Ist $\lambda_i$ eine komplexe Nullstelle, so ist $\lambda_i$ der Form $\lambda_i = a + i \cdot b$.
Zu jeder komplexen Nullstelle gibt es auch eine konjugierte Nullstelle: $\lambda_k = a - i \cdot b$.
Aus diesem Grund lösen für die komplexe Nullstelle ($\lambda_i$) $e^{x (a + ib)}$ und
$e^{x (a - ib)}$ das homogene DGL. \underline{Achtung:} Nachfolgend lohnt es sich oft die
Eulersche Identität zu verwenden: $e^{i \cdot x} = \cos(x) + i \sin(x)$.

Die allgemeine Lösung der homogenen DGL haben wir nun gefunden. Die Unbekannten $C_i$
können gefunden werden, wenn genügend Punkte gegeben sind, an denen der Funktionswert bekannt ist.

Hat man ursprünglich eine inhomogene DGL vorliegen, so muss man für die allgemeine Lösung
noch die partikuläre Lösung des inhomogenen DGL berechnen. Dazu werden die Unbekannten
$C_i$ durch Funktionen $u_i(x)$ ersetzt. So wird aus
$y_h = C_2 x e^{\lambda_1 x} + C_1 e^{\lambda_1 x} \Rightarrow
y_p = u_2(x) x e^{\lambda_1 x} + u_1(x) e^{\lambda_1 x}$ (eine doppelte Nullstelle).

Jetzt geht es darum die Funktionen $u_i(x)$ zu bestimmen, um sie in die vorherige
$y_p$-Gleichung einsetzen zu können. Dazu stellen wir $i$ Gleichungen auf.
Also so viele, wie wir unbekannte Funktionen $u_i(x)$ haben:
\begin{align*}
u_2(x)' (x e^{\lambda_1 x}) + u_1(x)' (e^{\lambda_1 x}) &= 0\\
u_2(x)' (x e^{\lambda_1 x})' + u_1(x)' (e^{\lambda_1 x})' &= g(x)
\end{align*}

Das Prinzip ist folgendes: Bis auf die letzte Gleichung, wird gleich $0$ gesetzt.
Die letzte Gleichung wird gleich $g(x)$ gesetzt.
Unsere unbekannten Funktionen werden jeweils einmal abgeleitet, egal in welcher
Gleichung wir sind. Pro Zeile, die man weiter runter geht, wird der Term mit $e^{\lambda_i x}$
jeweils einmal mehr abgeleitet. In der ersten Zeile wird zum Beispiel $e^{\lambda_1 x}$ nicht abgeleitet,
in der nächsten Gleichung wird es einmal abgeleitet. Hätten wir mehr Unbekannte Funktionen,
so würde in der folgenden Zeile zwei mal abgeleitet werden. Im Allgemeinen gilt also:
{\footnotesize
\begin{align*}
u_1(x)' y_{h1}(x) + u_2(x)' y_{h2}(x) + \ldots + u_n(x)' y_{hn}(x) &= 0\\
u_1(x)' y_{h1}(x)' + u_2(x)' y_{h2}(x)' + \ldots + u_n(x)' y_{hn}(x)' &= 0\\
u_1(x)' y_{h1}(x)'' + u_2(x)' y_{h2}(x)'' + \ldots + u_n(x)' y_{hn}(x)'' &= 0\\
&\ldots\\
u_1(x)' y_{h1}(x)^{(n-1)} + u_2(x)' y_{h2}(x)^{(n-1)} + \ldots + u_n(x)' y_{hn}(x)^{(n-1)} &= g(x)
\end{align*}
}

Diese Gleichungen werden nun jeweils aufgelöst, bis man $u_i(x)$ erhält. Um zu
$u_i(x)$ zu gelangen, muss auf dem Weg einmal die Gleichung auf beiden Seiten
integriert werden. Hat man alle $u_i(x)$, so setzt man diese in unsere
ursprüngliche $y_p$ Gleichung ein.

Nun kann die allgemeine Lösung des inhomogenen DGL berechnet werden. Dazu
summiert man $y_h$ und $y_p$: $y = y_h + y_p$. Dies ist die allgemeine Lösung.
Hat man konkrete Punkte, an denen die Funktion ausgewertet wurde, so kann man
die Unbekannten $C_i$ berechnen.

\subsubsection{Beispiel}
Es soll $y'' + y = \frac{2}{\cos(x)}$ ausgerechnet werden.

Zuerst sehen wir uns das homogene DGL an:
\begin{align*}
y'' + y &= 0\\
\Rightarrow \lambda^2 e^{\lambda x} + e^{\lambda x} &= 0\\
\Leftrightarrow e^{\lambda x} (\lambda^2 + 1) &=0 \\
\Rightarrow \lambda^2 + 1 &= 0\\
\Leftrightarrow \lambda^2 &= -1 \quad
\Rightarrow \underline{\lambda_1 = i},\, \underline{\lambda_2 = -i}
\end{align*}

Somit ist die allgemeine Lösung des homogenen DGL:
\begin{align*}
y_h &= C_1 \cdot e^{ix} + C_2 \cdot e^{-ix}\\
&= C_1 (\cos(x) + i\sin(x)) + C_2(\cos(x) - i\sin(x))\\
&= \sin(x) \underbrace{(i C_1 + i C_2)}_{ = D_1} + \cos(x) \underbrace{(C_1 + C_2)}_{= D_2}\\
&= \underline{D_1 \sin(x) + D_2 \cos(x) = y_h}
\end{align*}

Da es sich um ein inhomogenes DGL handelt, berechnen wir als nächstes die partikuläre
Lösung:
\begin{align*}
y_p &= \underbrace{u_1(x)}_{D_1 \text{ in } y_h} \sin(x) + \underbrace{u_2(x)}_{D_2 \text{ in } y_h} \cos(x)\\
&\Rightarrow \left|
	\begin{aligned}
		u_1(x)' \sin(x) + u_2(x)' \cos(x) &= 0\\
		u_1(x)' \sin(x)' + u_2(x)' \cos(x)' &= \frac{2}{\cos(x)}
	\end{aligned}
\right|\\
&= \ldots\\
&\Rightarrow u_1(x)' = -2 \tan(x),\, u_2(x)' = 2\\
&\Leftrightarrow u_1(x) = 2 \ln(\cos(x)),\, u_2(x) = 2x\\
&\Rightarrow \underline{y_p = 2 \ln(\cos(x)) \sin(x) + 2x \cos(x)}
\end{align*}

Da wir nun auch die partikuläre Lösung haben, können wir die allgemeine Lösung
des inhomogenen DGL berechnen:
$\underline{\underline{y}} = y_h + y_p = \underline{\underline{D_1 \sin(x) + D_2 \cos(x) + 2 \ln(\cos(x)) \sin(x) + 2x \cos(x)}}$
