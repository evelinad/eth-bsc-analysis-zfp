\onecolumn
\begin{landscape}
\section{Reihen Tabellen}

\begin{tabular}{|p{5cm}||p{5cm}|p{4cm}|p{4cm}|p{4cm}|p{4cm}|}
\hline & \multicolumn{4}{l}{schnelles Fallen} & langsames Fallen \\
\hline
 
wie schnell gehen die $a_n$ gegen 0 & exponentiell wie $q^n, |q| < 1$ &
polynominal wie $n^{-\alpha}, \alpha > 1$ & \multicolumn{2}{|c|}{höchstens wie
$1/n$} & gar nicht \\ \hline

Beispiele & \begin{align*} a_n = \frac{n^8}{2^n},\\ a_n =
(\sqrt[n]{n} - 1)^n,\\
a_n = \frac{1}{n!},\\ a_n = \left( \frac{-1}{4}\right)^n \end{align*} &
\begin{align*} a_n = \frac{1}{n^2}, \\ a_n = \frac{1}{n^{100}}, \\ a_n =
\frac{1}{(n + \ln n)^2}, \\ a_n = \frac{20}{n^2 - 33} \end{align*} &
\begin{align*} a_n = \frac{(-1)^n}{\ln n}, \\ a_n = \frac{(-1)^n}{n}
\end{align*} & \begin{align*} a_n = \frac{1}{\ln n}, \\ a_n = \frac{1}{n + \ln
n}, \\ a_n = \frac{1}{n} \end{align*} & \begin{align*} a_n = (-1)^n, \\ a_n =
\sin n, \\ a_n = n^2 \end{align*}\\ \hline

passende Konvergenzkriterien & Wurzel- und Quotientenkriterium & Integral- und
Verdichtungskriterium & Leibniz-Kriterium & & $a_n \not\to 0$ \\ \hline

Vergleichs-, Majoranten-, Minorantenkriterium & Vergleichen mit $q^n$ &
Vergleichen mit $n^{-\alpha}$ & kein Vergleich möglich & Vergleichen mit
$\frac{1}{n}$ & \\ \hline

Konvergenz-verhalten & \multicolumn{2}{c|}{absolute Konvergenz} & keine
absolute Konvergenz (einfach Konvergenz) & \multicolumn{2}{|c|}{Divergenz} \\
\hline

\end{tabular}

\begin{tabular}{|p{4cm}|p{15cm}|}\hline
\multicolumn{2}{|c|}{direkte Kriterien}\\ \hline

\textbf{Quotientenkriterium} & Gut für Reihen, die Fakultäten oder Glieder der
Form $a^n$ enthalten. Nicht auf Reihen anwendbar, in denen die Glieder nur wie
eine Potenz von $n$ fallen. \\ \hline

\textbf{Wurzelkriterium} & Gut in Reihen, deren Glieder $n$-te Potenzen sind,
zusammen mit der Stirlingformel oft auch bei Fakultäten anwendbar. \\ \hline

\textbf{Leibnizkriterium} & Nur für alternierende Reihen.\\ \hline

\textbf{Integralkriterium} & Anwendbar auf monotone Reihen.\\ \hline

\multicolumn{2}{|c|}{direkte Kriterien}\\ \hline

\textbf{Vergleichskriterium} & Ermöglicht es "`Störterme"' wegzulassen und so
einfachere Reihen zu untersuchen \\ \hline

\textbf{Verdichtungskriterium} & Bei monotonen Reihen anwendbar. Für Reihen mit
langsam fallenden Gliedern \\ \hline

\textbf{Majoranten- und Minorantenkriterium} & Ähnlich wie Vergleichskriterium.
Wird mit einer Reihe verglichen, deren Glieder stets kleiner oder grösser sind.
\\ \hline
\end{tabular}

\end{landscape}
\twocolumn