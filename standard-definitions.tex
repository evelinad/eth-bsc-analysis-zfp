%%
%% 
%% (C) 2011 Gregor Wegberg
%%
%% Based on work of Stefan Heule, Licensed as Creative Commons Attribution-Share Alike 3.0 Unported
%% 
%% License: Creative Commons Attribution-Share Alike 3.0 Unported
%% http://creativecommons.org/licenses/by-sa/3.0/
%% 

%% The following lines need to be included in the main tex file
%\documentclass[portrait,a4paper,titlepage]{article}
%%%
%% 
%% (C) 2011 Gregor Wegberg
%%
%% Based on work of Stefan Heule, Licensed as Creative Commons Attribution-Share Alike 3.0 Unported
%% 
%% License: Creative Commons Attribution-Share Alike 3.0 Unported
%% http://creativecommons.org/licenses/by-sa/3.0/
%% 

%% The following lines need to be included in the main tex file
%\documentclass[portrait,a4paper,titlepage]{article}
%%%
%% 
%% (C) 2011 Gregor Wegberg
%%
%% Based on work of Stefan Heule, Licensed as Creative Commons Attribution-Share Alike 3.0 Unported
%% 
%% License: Creative Commons Attribution-Share Alike 3.0 Unported
%% http://creativecommons.org/licenses/by-sa/3.0/
%% 

%% The following lines need to be included in the main tex file
%\documentclass[portrait,a4paper,titlepage]{article}
%%%
%% 
%% (C) 2011 Gregor Wegberg
%%
%% Based on work of Stefan Heule, Licensed as Creative Commons Attribution-Share Alike 3.0 Unported
%% 
%% License: Creative Commons Attribution-Share Alike 3.0 Unported
%% http://creativecommons.org/licenses/by-sa/3.0/
%% 

%% The following lines need to be included in the main tex file
%\documentclass[portrait,a4paper,titlepage]{article}
%\input{standard-definitions.tex}

\usepackage[utf8]{inputenc}
\usepackage{fontenc}

\usepackage{color}
\usepackage{soul}
\usepackage{soulutf8}

% Stichwortverzeichnis
\usepackage{makeidx}
%\usepackage{idxlayout}

\usepackage{alltt}
\renewcommand{\ttdefault}{txtt}

\usepackage{amssymb,amsfonts,amsmath}
\usepackage[e]{esvect}

\usepackage{algorithmicx}

\usepackage[pdftex]{graphicx}
\usepackage{epstopdf}
\usepackage[svgnames]{xcolor}

\usepackage{cancel}

\usepackage{pgf,tikz}
\usetikzlibrary{arrows}

\usepackage{geometry}
\geometry{a4paper, left=20mm, right=20mm, top=25mm, bottom=20mm}

% Allows fancy stuff in the page header
\usepackage{fancyhdr}
\pagestyle{fancy}

% hyperref
\usepackage[colorlinks=false,pdfborder={0 0 0}]{hyperref}

% multirow and multicol
\usepackage{multirow}
\usepackage{multicol}
\columnsep24pt
\columnseprule0.1pt

% enumerate
\renewcommand\theenumi{\arabic{enumi}}
\renewcommand\labelenumi{\theenumi.}
\renewcommand\theenumii{\roman{enumii}}
\renewcommand\labelenumii{\theenumii)}

\usepackage{listings}
\lstset{
    floatplacement={tbp}
    basicstyle=\ttfamily\mdseries,
    identifierstyle=,
    stringstyle=\color{gray},
    numbers=left,
    numbersep=5pt,
    inputencoding=utf8,
    xleftmargin=8pt,
    xrightmargin=8pt,
    keywordstyle=[1]\bfseries,
    keywordstyle=[2]\bfseries,
    keywordstyle=[3]\bfseries,
    keywordstyle=[4]\bfseries,
    numberstyle=\tiny,
    stepnumber=1,
    breaklines=true,
    frame=lines,
    showstringspaces=false,
    tabsize=2,
    commentstyle=\color{gray},
    captionpos=b,
    float=float,
    language={Java}
}
\newcommand{\code}[1]{\lstinline{#1}}

% depth of section numbering
\setcounter{secnumdepth}{4}

%% Redefine the \paragraph command:
\makeatletter
\renewcommand\paragraph{\@startsection{paragraph}{4}{0mm}%
    {-\baselineskip}%
    {0.5\baselineskip}%
    {\normalfont\bfseries}%
}%
\makeatother

% parindent
\parindent0px
\parskip3pt

% redefine greek letters
\renewcommand{\phi}{\varphi}
\renewcommand{\epsilon}{\varepsilon}

% shortcuts in math mode
\newcommand{\bs}{\boldsymbol}
\newcommand{\mc}{\mathcal}
\newcommand{\norm}[1]{| \!\:\! | #1 | \!\:\! |}
\newcommand{\with}{\;|\;} % with in set notation
\newcommand{\ds}{\displaystyle}
\newcommand{\nop}[1]{}
\newcommand{\argmax}{\operatorname*{arg\;max}}
\newcommand{\argmin}{\operatorname*{arg\;min}}
\newcommand{\rmd}{\mathrm{d}} % for integrals
\newcommand{\ggT}{\operatorname*{ggT}}
\newcommand{\kgV}{\operatorname*{kgV}}
\newcommand{\id}{\operatorname{id}}
\newcommand{\grad}{\operatorname{grad}}
\newcommand{\rot}{\operatorname{rot}}

% number sets
\newcommand{\R}{\mathbb{R}}
\newcommand{\Z}{\mathbb{Z}}
\newcommand{\N}{\mathbb{N}}
\newcommand{\Q}{\mathbb{Q}}
\newcommand{\C}{\mathbb{C}}
\newcommand{\F}{\mathbb{F}}
\newcommand{\E}{\mathbb{E}}
\newcommand{\LL}{\mathcal{L}}
\newcommand{\powerset}{\mathcal P}

% probabilities
\newcommand{\Prob}[1]{\operatorname{Pr}\left[#1\right]}
\newcommand{\Ex}[1]{\mathbb{E}\left[#1\right]}

% todo
\newcommand{\todo}[1]{\sethlcolor{red}\hl{$\ggg$ \textbf{TODO} [#1]}\sethlcolor{yellow}}


% big-o notation
\newcommand{\bigO}[1]{\mc O\left(#1\right)}




\usepackage[utf8]{inputenc}
\usepackage{fontenc}

\usepackage{color}
\usepackage{soul}
\usepackage{soulutf8}

% Stichwortverzeichnis
\usepackage{makeidx}
%\usepackage{idxlayout}

\usepackage{alltt}
\renewcommand{\ttdefault}{txtt}

\usepackage{amssymb,amsfonts,amsmath}
\usepackage[e]{esvect}

\usepackage{algorithmicx}

\usepackage[pdftex]{graphicx}
\usepackage{epstopdf}
\usepackage[svgnames]{xcolor}

\usepackage{cancel}

\usepackage{pgf,tikz}
\usetikzlibrary{arrows}

\usepackage{geometry}
\geometry{a4paper, left=20mm, right=20mm, top=25mm, bottom=20mm}

% Allows fancy stuff in the page header
\usepackage{fancyhdr}
\pagestyle{fancy}

% hyperref
\usepackage[colorlinks=false,pdfborder={0 0 0}]{hyperref}

% multirow and multicol
\usepackage{multirow}
\usepackage{multicol}
\columnsep24pt
\columnseprule0.1pt

% enumerate
\renewcommand\theenumi{\arabic{enumi}}
\renewcommand\labelenumi{\theenumi.}
\renewcommand\theenumii{\roman{enumii}}
\renewcommand\labelenumii{\theenumii)}

\usepackage{listings}
\lstset{
    floatplacement={tbp}
    basicstyle=\ttfamily\mdseries,
    identifierstyle=,
    stringstyle=\color{gray},
    numbers=left,
    numbersep=5pt,
    inputencoding=utf8,
    xleftmargin=8pt,
    xrightmargin=8pt,
    keywordstyle=[1]\bfseries,
    keywordstyle=[2]\bfseries,
    keywordstyle=[3]\bfseries,
    keywordstyle=[4]\bfseries,
    numberstyle=\tiny,
    stepnumber=1,
    breaklines=true,
    frame=lines,
    showstringspaces=false,
    tabsize=2,
    commentstyle=\color{gray},
    captionpos=b,
    float=float,
    language={Java}
}
\newcommand{\code}[1]{\lstinline{#1}}

% depth of section numbering
\setcounter{secnumdepth}{4}

%% Redefine the \paragraph command:
\makeatletter
\renewcommand\paragraph{\@startsection{paragraph}{4}{0mm}%
    {-\baselineskip}%
    {0.5\baselineskip}%
    {\normalfont\bfseries}%
}%
\makeatother

% parindent
\parindent0px
\parskip3pt

% redefine greek letters
\renewcommand{\phi}{\varphi}
\renewcommand{\epsilon}{\varepsilon}

% shortcuts in math mode
\newcommand{\bs}{\boldsymbol}
\newcommand{\mc}{\mathcal}
\newcommand{\norm}[1]{| \!\:\! | #1 | \!\:\! |}
\newcommand{\with}{\;|\;} % with in set notation
\newcommand{\ds}{\displaystyle}
\newcommand{\nop}[1]{}
\newcommand{\argmax}{\operatorname*{arg\;max}}
\newcommand{\argmin}{\operatorname*{arg\;min}}
\newcommand{\rmd}{\mathrm{d}} % for integrals
\newcommand{\ggT}{\operatorname*{ggT}}
\newcommand{\kgV}{\operatorname*{kgV}}
\newcommand{\id}{\operatorname{id}}
\newcommand{\grad}{\operatorname{grad}}
\newcommand{\rot}{\operatorname{rot}}

% number sets
\newcommand{\R}{\mathbb{R}}
\newcommand{\Z}{\mathbb{Z}}
\newcommand{\N}{\mathbb{N}}
\newcommand{\Q}{\mathbb{Q}}
\newcommand{\C}{\mathbb{C}}
\newcommand{\F}{\mathbb{F}}
\newcommand{\E}{\mathbb{E}}
\newcommand{\LL}{\mathcal{L}}
\newcommand{\powerset}{\mathcal P}

% probabilities
\newcommand{\Prob}[1]{\operatorname{Pr}\left[#1\right]}
\newcommand{\Ex}[1]{\mathbb{E}\left[#1\right]}

% todo
\newcommand{\todo}[1]{\sethlcolor{red}\hl{$\ggg$ \textbf{TODO} [#1]}\sethlcolor{yellow}}


% big-o notation
\newcommand{\bigO}[1]{\mc O\left(#1\right)}




\usepackage[utf8]{inputenc}
\usepackage{fontenc}

\usepackage{color}
\usepackage{soul}
\usepackage{soulutf8}

% Stichwortverzeichnis
\usepackage{makeidx}
%\usepackage{idxlayout}

\usepackage{alltt}
\renewcommand{\ttdefault}{txtt}

\usepackage{amssymb,amsfonts,amsmath}
\usepackage[e]{esvect}

\usepackage{algorithmicx}

\usepackage[pdftex]{graphicx}
\usepackage{epstopdf}
\usepackage[svgnames]{xcolor}

\usepackage{cancel}

\usepackage{pgf,tikz}
\usetikzlibrary{arrows}

\usepackage{geometry}
\geometry{a4paper, left=20mm, right=20mm, top=25mm, bottom=20mm}

% Allows fancy stuff in the page header
\usepackage{fancyhdr}
\pagestyle{fancy}

% hyperref
\usepackage[colorlinks=false,pdfborder={0 0 0}]{hyperref}

% multirow and multicol
\usepackage{multirow}
\usepackage{multicol}
\columnsep24pt
\columnseprule0.1pt

% enumerate
\renewcommand\theenumi{\arabic{enumi}}
\renewcommand\labelenumi{\theenumi.}
\renewcommand\theenumii{\roman{enumii}}
\renewcommand\labelenumii{\theenumii)}

\usepackage{listings}
\lstset{
    floatplacement={tbp}
    basicstyle=\ttfamily\mdseries,
    identifierstyle=,
    stringstyle=\color{gray},
    numbers=left,
    numbersep=5pt,
    inputencoding=utf8,
    xleftmargin=8pt,
    xrightmargin=8pt,
    keywordstyle=[1]\bfseries,
    keywordstyle=[2]\bfseries,
    keywordstyle=[3]\bfseries,
    keywordstyle=[4]\bfseries,
    numberstyle=\tiny,
    stepnumber=1,
    breaklines=true,
    frame=lines,
    showstringspaces=false,
    tabsize=2,
    commentstyle=\color{gray},
    captionpos=b,
    float=float,
    language={Java}
}
\newcommand{\code}[1]{\lstinline{#1}}

% depth of section numbering
\setcounter{secnumdepth}{4}

%% Redefine the \paragraph command:
\makeatletter
\renewcommand\paragraph{\@startsection{paragraph}{4}{0mm}%
    {-\baselineskip}%
    {0.5\baselineskip}%
    {\normalfont\bfseries}%
}%
\makeatother

% parindent
\parindent0px
\parskip3pt

% redefine greek letters
\renewcommand{\phi}{\varphi}
\renewcommand{\epsilon}{\varepsilon}

% shortcuts in math mode
\newcommand{\bs}{\boldsymbol}
\newcommand{\mc}{\mathcal}
\newcommand{\norm}[1]{| \!\:\! | #1 | \!\:\! |}
\newcommand{\with}{\;|\;} % with in set notation
\newcommand{\ds}{\displaystyle}
\newcommand{\nop}[1]{}
\newcommand{\argmax}{\operatorname*{arg\;max}}
\newcommand{\argmin}{\operatorname*{arg\;min}}
\newcommand{\rmd}{\mathrm{d}} % for integrals
\newcommand{\ggT}{\operatorname*{ggT}}
\newcommand{\kgV}{\operatorname*{kgV}}
\newcommand{\id}{\operatorname{id}}
\newcommand{\grad}{\operatorname{grad}}
\newcommand{\rot}{\operatorname{rot}}

% number sets
\newcommand{\R}{\mathbb{R}}
\newcommand{\Z}{\mathbb{Z}}
\newcommand{\N}{\mathbb{N}}
\newcommand{\Q}{\mathbb{Q}}
\newcommand{\C}{\mathbb{C}}
\newcommand{\F}{\mathbb{F}}
\newcommand{\E}{\mathbb{E}}
\newcommand{\LL}{\mathcal{L}}
\newcommand{\powerset}{\mathcal P}

% probabilities
\newcommand{\Prob}[1]{\operatorname{Pr}\left[#1\right]}
\newcommand{\Ex}[1]{\mathbb{E}\left[#1\right]}

% todo
\newcommand{\todo}[1]{\sethlcolor{red}\hl{$\ggg$ \textbf{TODO} [#1]}\sethlcolor{yellow}}


% big-o notation
\newcommand{\bigO}[1]{\mc O\left(#1\right)}




\usepackage[utf8]{inputenc}
\usepackage{fontenc}

\usepackage{color}
\usepackage{soul}
\usepackage{soulutf8}

% Stichwortverzeichnis
\usepackage{makeidx}


\usepackage{alltt}
\renewcommand{\ttdefault}{txtt}

\usepackage{amssymb,amsfonts,amsmath}
\usepackage[e]{esvect}

\usepackage{algorithmicx}

\usepackage[pdftex]{graphicx}
\usepackage{epstopdf}
\usepackage[svgnames]{xcolor}

\usepackage{cancel}

\usepackage{pgf,tikz}
\usetikzlibrary{arrows}

\usepackage{geometry}
\geometry{a4paper, left=20mm, right=20mm, top=25mm, bottom=20mm}

% Allows fancy stuff in the page header
\usepackage{fancyhdr}
\pagestyle{fancy}

% hyperref
\usepackage[colorlinks=false,pdfborder={0 0 0}]{hyperref}

% multirow and multicol
\usepackage{multirow}
\usepackage{multicol}
\columnsep24pt
\columnseprule0.1pt

% enumerate
\renewcommand\theenumi{\arabic{enumi}}
\renewcommand\labelenumi{\theenumi.}
\renewcommand\theenumii{\roman{enumii}}
\renewcommand\labelenumii{\theenumii)}

\usepackage{listings}
\lstset{
    floatplacement={tbp}
    basicstyle=\ttfamily\mdseries,
    identifierstyle=,
    stringstyle=\color{gray},
    numbers=left,
    numbersep=5pt,
    inputencoding=utf8,
    xleftmargin=8pt,
    xrightmargin=8pt,
    keywordstyle=[1]\bfseries,
    keywordstyle=[2]\bfseries,
    keywordstyle=[3]\bfseries,
    keywordstyle=[4]\bfseries,
    numberstyle=\tiny,
    stepnumber=1,
    breaklines=true,
    frame=lines,
    showstringspaces=false,
    tabsize=2,
    commentstyle=\color{gray},
    captionpos=b,
    float=float,
    language={Java}
}
\newcommand{\code}[1]{\lstinline{#1}}

% depth of section numbering
\setcounter{secnumdepth}{4}

%% Redefine the \paragraph command:
\makeatletter
\renewcommand\paragraph{\@startsection{paragraph}{4}{0mm}%
    {-\baselineskip}%
    {0.5\baselineskip}%
    {\normalfont\bfseries}%
}%
\makeatother

% parindent
\parindent0px
\parskip3pt

% redefine greek letters
\renewcommand{\phi}{\varphi}
\renewcommand{\epsilon}{\varepsilon}

% shortcuts in math mode
\newcommand{\bs}{\boldsymbol}
\newcommand{\mc}{\mathcal}
\newcommand{\norm}[1]{| \!\:\! | #1 | \!\:\! |}
\newcommand{\with}{\;|\;} % with in set notation
\newcommand{\ds}{\displaystyle}
\newcommand{\nop}[1]{}
\newcommand{\argmax}{\operatorname*{arg\;max}}
\newcommand{\argmin}{\operatorname*{arg\;min}}
\newcommand{\rmd}{\mathrm{d}} % for integrals
\newcommand{\ggT}{\operatorname*{ggT}}
\newcommand{\kgV}{\operatorname*{kgV}}
\newcommand{\id}{\operatorname{id}}

% number sets
\newcommand{\R}{\mathbb{R}}
\newcommand{\Z}{\mathbb{Z}}
\newcommand{\N}{\mathbb{N}}
\newcommand{\Q}{\mathbb{Q}}
\newcommand{\C}{\mathbb{C}}
\newcommand{\F}{\mathbb{F}}
\newcommand{\E}{\mathbb{E}}
\newcommand{\LL}{\mathcal{L}}
\newcommand{\powerset}{\mathcal P}

% probabilities
\newcommand{\Prob}[1]{\operatorname{Pr}\left[#1\right]}
\newcommand{\Ex}[1]{\mathbb{E}\left[#1\right]}

% todo
\newcommand{\todo}[1]{\sethlcolor{red}\hl{$\ggg$ \textbf{TODO} [#1]}\sethlcolor{yellow}}


% big-o notation
\newcommand{\bigO}[1]{\mc O\left(#1\right)}


