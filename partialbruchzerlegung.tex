\section{Partialbruchzerlegung}
Mit dieser Methode wird ein schwieriger Bruch in eine Summe von einfacheren
Brüchen zerlegt. Damit lässt sich der ursprüngliche Bruch einfacher integrieren.

\begin{enumerate}
	\item Den Grad des Zählers und des Nenners vergleichen von $R$
	\begin{enumerate}
		\item Ist der Zählergrad (über Bruchstrich) grösser oder gleich dem Nennergrad
		(unter Bruchstrich), so dividiert man den Zähler durch den Nenner. Man erhält
		daraus das Polynom $P^*$ und möglicherweise einen Rest $R$, sodass gilt: $R =
		P + R^*$.
		\item Ist $R^* \equiv 0$, so ist das Verfahren abgeschlossen. Sonst arbeitet
		man nun mit dem Rest $R^*$ weiter als Bruch, den man ansieht.
	\end{enumerate}
	\item Man berechnet die Nullstellen vom Nenner des Bruches.
	\item Nun setzt man den ursprünglichen Bruch gleich der Summe der
	Partialbrüche. Die Partialbrüche sind jeweils abhängig von den Nullstellen:
	\begin{enumerate}
		\item Für jede einfache reelle Nullstelle $x_i$ ist der Summand
		$\frac{a_{i1}}{x-x_i}$ zu nehmen
		\item Für jede $r_i$-fache Nullstelle $x_i$ erhält man $r_i$ Summanden:
		$\frac{a_{i1}}{x-x_i} + \frac{a_{i2}}{(x-x_i)^2} + \ldots +
		\frac{a_{ir_i}}{(x-x_i)^{r_i}}$
	\end{enumerate}
	\item Nun berechnet man die unbekannten $a_{ij}$ indem man die Partialbrüche
	gleichnamig macht und dann die Koeffizienten des ursprünglichen Zählers mit
	denen des gleichnamigen Bruchs vergleicht.
\end{enumerate}

\subsection{Beispiel}
$R(x) = \frac{x^2}{x^2-2x+1}$.

Der Zählergrad ist gleich dem Nennergrad,
weswegen wir eine Polynomdivision durchfüren: $\Rightarrow R(x) = 1 +
\frac{2x-1}{(x-1)^2}$.

Aus $(x-1)^2$ folgt, das wir nur eine Nullstelle haben $x_0 = 1$. Es handelt
sich dabei um eine doppelte Nullstelle.

Somit gilt:
\begin{align*}
\frac{2x-1}{(x-1)^2} &= \frac{a_1}{x-1} + \frac{a_2}{(x-1)^2}\\
2x-1 &= a_1(x-1) + a_2\\
2x-1 &= a_1 x - a_1 + a_2
\end{align*}
Daraus folgt, dass $a_1 = 2$ und $a_2 = 1$ (lineares Gleichungssystem).

Somit gilt: $R(x) = \frac{x^2}{x^2-2x+1} = 1 + \frac{2}{x-1} +
\frac{1}{(x-1)^2}$
